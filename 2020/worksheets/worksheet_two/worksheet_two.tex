% Created 2020-06-17 Wed 17:59
% Intended LaTeX compiler: pdflatex
\documentclass[11pt]{article}
\usepackage[utf8]{inputenc}
\usepackage[T1]{fontenc}
\usepackage{graphicx}
\usepackage{grffile}
\usepackage{longtable}
\usepackage{wrapfig}
\usepackage{rotating}
\usepackage[normalem]{ulem}
\usepackage{amsmath}
\usepackage{textcomp}
\usepackage{amssymb}
\usepackage{capt-of}
\usepackage{hyperref}
\author{Rodda John}
\date{(expected 30 minutes completion time)}
\title{Hypothekids Python Worksheet 2}
\hypersetup{
 pdfauthor={Rodda John},
 pdftitle={Hypothekids Python Worksheet 2},
 pdfkeywords={},
 pdfsubject={},
 pdfcreator={Emacs 25.2.1 (Org mode N/A)}, 
 pdflang={English}}
\begin{document}

\maketitle

\section{A Loop}
\label{sec:orgf3bea5e}
This is a new concept, introduced through old concepts!  You will need to read through the "supplemental information", but the problem is explained first.

This likely won't make much sense, but the official documentation can be found \href{https://docs.python.org/3/reference/compound\_stmts.html\#the-while-statement}{here}.
\subsection{The Problem}
\label{sec:org2026065}
Write a piece of code that prints all the \textbf{odd} numbers from 1 to 1000.
\subsection{Sample Output}
\label{sec:orgf1bc1b3}
If the code did this for the numbers from 1 to 10, it would print out:
\begin{itemize}
\item 1
\item 3
\item 5
\item 7
\item 9
\end{itemize}
\subsection{Supplemental Information}
\label{sec:org33064ec}
An \texttt{if} statement takes the form \texttt{if <boolean expression:} followed by indented code.  This code is conditionally executed if and only if the boolean condition is true.

A \texttt{while} statement takes the form \texttt{while <boolean expression>:} followed by indented code.  This code is executed \textbf{while} the boolean condition is true.

Thus, the condition should start out true and change within the block so as to make it false.

For example, the following code prints all the numbers from 1 to 100.

\begin{verbatim}
i = 1

while i < 100:
    print (i)
    i = i + 1
\end{verbatim}
\section{Nested Loops}
\label{sec:org82fac87}
\subsection{The Problem}
\label{sec:orgc065df9}
Write a piece of code that renders all the coordinates on a 100 by 100 grid.
\subsection{Sample Output}
\label{sec:org26a31f0}
If the code did this on a 2 by 2 grid, it would print out:
\begin{itemize}
\item (0, 0)
\item (0, 1)
\item (1, 0)
\item (1, 1)
\end{itemize}

Hint: In the previous problem, you had one while loop with one moving variable, here you will need two.
\end{document}
