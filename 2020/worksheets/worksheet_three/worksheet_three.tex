% Created 2020-06-13 Sat 10:09
% Intended LaTeX compiler: pdflatex
\documentclass[11pt]{article}
\usepackage[utf8]{inputenc}
\usepackage[T1]{fontenc}
\usepackage{graphicx}
\usepackage{grffile}
\usepackage{longtable}
\usepackage{wrapfig}
\usepackage{rotating}
\usepackage[normalem]{ulem}
\usepackage{amsmath}
\usepackage{textcomp}
\usepackage{amssymb}
\usepackage{capt-of}
\usepackage{hyperref}
\author{Rodda John}
\date{(expected 20 minutes completion time)}
\title{Hypothekids Python Worksheet 3}
\hypersetup{
 pdfauthor={Rodda John},
 pdftitle={Hypothekids Python Worksheet 3},
 pdfkeywords={},
 pdfsubject={},
 pdfcreator={Emacs 25.2.1 (Org mode N/A)}, 
 pdflang={English}}
\begin{document}

\maketitle

\section{Lists}
\label{sec:org7418924}
A list is a new data type.  Instead of holding a single value, it holds multiple values.

You've actually already been introduced to them, and just not known it.  A string is simply a list of characters.

\subsection{How to interact with lists}
\label{sec:org10243bf}
Lists can be defined like this \texttt{l = [1, 2, 3, 4, 5]}.  This is a list of numbers.

All lists are ZERO INDEXED, ie, to access the first element of the list (\texttt{1}), you would write \texttt{l[0]}.  \texttt{l} being the name of the list, and \texttt{0} being the element you are trying to access.

\subsection{A samples}
\label{sec:org6aae5cd}

\begin{verbatim}
l = ['a', 'b', 'c', 'd', 'e', 'f'] # A list with 6 strings

print (l[0]) # This will print 'a'
print (l[2]) # This will print 'c'

print (l[5]) # This will print 'f'
\end{verbatim}

\subsection{Strings as lists}
\label{sec:org852edd5}
\begin{verbatim}
sample_string = 'this is a sample string'

print (sample_string[0]) # This will print 't'
print (sample_string[4]) # this will print ' '
\end{verbatim}

\subsection{Helper function}
\label{sec:org89c855b}
The \texttt{len(<iterable>)} function returns the length of a string.  See \href{https://docs.python.org/3/library/functions.html\#len}{the documentation} for more information.

We can use this function to iterate through a list:

\begin{verbatim}
l = [100, 110, 120, 130, 140, 150]

index = 0

while index < len(l):
    print (l[index])
    index = index + 1
\end{verbatim}

This will print out each value of the list.

Here's a sample way to sum all elements of a list:

\begin{verbatim}
l = [100, 110, 120, 130, 140, 150]

index = 0
sum = 0

while index < len(l):
    sum = sum + l[index]
    index = index + 1

print (sum)
\end{verbatim}

\subsection{The Problem:}
\label{sec:orgfa767b0}
Write a function that finds the average of a list of numbers.

ie, given the list of \texttt{[1, 2, 3, 4, 5]}, it should print \texttt{3}.

Try on \texttt{[14, 16, 15, 17, 19, 20, 23, 25]}, it should be 18.625.

Submit the code as usual through Google Classroom.
\end{document}
